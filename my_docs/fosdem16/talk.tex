%* installation (deps, build/install procedure, making module/ml work)
%* configuration (either via ./configure at build time, or via 
%   $LMOD_FOO), most important configure options
%* setting up the system spider cache: configuring Lmod, updating spider 
%  cache, difference between system/user spider cache
%* + all (other) aspects mentioned on slide #26 in this slide deck: 
%   explained + example :)
%* what settarg is all about

\documentclass{beamer}

\usetheme[headernav]{TACC} %%Drop the 'headernav' if you don't like
                           %%the stuff at the top of your slide

\usepackage{amsmath,amssymb,amsthm}
\usepackage{alltt}
\usepackage{graphicx}

\title{XALT: User Environment Tracking}


\author{Robert McLay, Mark Fahey, Reuben Budiardja, Sandra Sweat}
\institute{The Texas Advanced Computing Center, Argonne National Labs, NICS}

\date{Jan. 31, 2016}  %% Use this if you want to fix the date in
                      %% stone rather than use \today

\newcommand{\bfnabla}{\mbox{\boldmath$\nabla$}}
\newcommand{\laplacian}[1]{\bfnabla^2 #1}
\newcommand{\grad}[1]{\bfnabla #1}
\newcommand{\tgrad}[1]{\bfnabla^T #1}
\newcommand{\dvg}[1]{\bfnabla \cdot #1}
\newcommand{\curl}[1]{\bfnabla \times #1}
\newcommand{\lap}[1]{\bfnabla^2 #1}

\begin{document}

\begin{frame}
  \titlepage
\end{frame}

\section{XALT}

\begin{frame}{XALT: What runs on the system}
  \begin{itemize}
    \item A U.S. NSF Funded project: PI: Mark Fahey and Robert McLay
    \item A Census of what programs and libraries are run
    \item Running at TACC, NICS, U. Florida, KAUST, ...
    \item Integrates with TACC-Stats.
  \end{itemize}
\end{frame}

\begin{frame}{Design Goals}
  \begin{itemize}
    \item Be extremely light-weight
    \item Provide provenance data: How?
    \item How many use a library or application?
    \item Collect Data into a Database for analysis.
  \end{itemize}
\end{frame}

\begin{frame}{Design: Linker }
  \begin{itemize}
    \item The linker (ld) wrapper intercepts the user link line.
      \begin{itemize}
        \item A shell script wrapper, \texttt{ld} which uses python
          scripts
        \item Generate assembly code: key-value pairs
        \item Capture tracemap output from ld
        \item Transmit collected data in *.json format
      \end{itemize}
  \end{itemize}
\end{frame}

\begin{frame}{Design: Launcher}
  \begin{itemize}
    \item Program Launcher: mpirun, aprun, ibrun ...
      \begin{itemize}
        \item A shell script wrapper is called which uses python scripts
        \item Find Executable by parsing command
        \item Collect executable info, shared libraries, env.
        \item Transmit collected data in *.json format
      \end{itemize}
    \item The future is now.  This is nolonger necessary!
  \end{itemize}
\end{frame}

\begin{frame}{Design: Transmission to DB}
  \begin{itemize}
    \item File: collect nightly
    \item Syslog: Use Syslog filtering
    \item Direct to DB.
  \end{itemize}
\end{frame}

\begin{frame}{Lmod to XALT connection}
  \begin{itemize}
    \item Lmod spider walks entire module tree.
    \item Can build A Reverse Map from paths to modules
    \item Can map program \& libraries to modules.
    \item /opt/apps/i15/mv2\_2\_1/phdf5/1.8.14/lib/libhdf5.so.9
      $\Rightarrow$ phdf5/1.8.14(intel/15.02:mvapich2/2.1)
  \end{itemize}
\end{frame}

\begin{frame}{Lmod: Priority Path}
  \begin{itemize}
    \item Fixed Job Launcher: ibrun, aprun
    \item Variable Launchers: mpirun, mpiexec
    \item Priority Path: \\
      \texttt{prepend\_path\{"PATH", "/opt/apps/xalt/1.0/bin", priority=100\}}
  \end{itemize}
\end{frame}

\begin{frame}[fragile]
    \frametitle{Database Changes (I)}
  \begin{itemize}
      \item Tables sizes in XALT:
 {\small
    \begin{alltt}
+------------------+------------+
| Table            | Size in MB |
+------------------+------------+
| join_run_env     |  199603.00 |
| join_run_object  |    9388.00 |
| join_link_object |    5013.00 |
| xalt_run         |    4613.00 |
| xalt_object      |    4175.00 |
| xalt_link        |     814.00 |
+------------------+------------+
    \end{alltt}
}
    \item \texttt{join\_run\_env} has 2.1 billion rows
  \end{itemize}
\end{frame}

\begin{frame}{Database Changes (II)}
  \begin{itemize}
    \item Environment variables are important.
    \item But mainly for reproducing results
    \item Not SQL tests (mostly)
  \end{itemize}
\end{frame}

\begin{frame}{Database Changes (III): New Design}
  \begin{itemize}
    \item Store complete env $\Rightarrow$ compressed json blob
    \item Filter Env's with Accept Test followed by Reject Test
    \item Instead of 250 vars per job $\Rightarrow$ 20 to 30.
  \end{itemize}
\end{frame}

\begin{frame}{Protecting XALT (I): UTF8 Characters}
  \begin{itemize}
    \item Linux supports UTF8 Characters in file names, env. vars.
    \item Python supports UTF8 if you know what you are doing.
    \item Switch XALT to use \texttt{cursor.execute(query, (job\_id,
        user, ...)}
    \item Where \texttt{query="INSERT INTO table VALUE(\%s,\%s)"}
    \item This prevent SQL injection: ``johnny drop tables;''
    \item Also supports UTF8 characters.
  \end{itemize}
\end{frame}

\begin{frame}{Protecting XALT (II): PYTHONHOME,...}
  \begin{itemize}
    \item Four Ways: LD\_LIBRARY\_PATH, PATH, PYTHONPATH, PYTHONHOME
    \item Solution: LD\_LIBRARY\_PATH="@ld\_lib\_path@" PATH= @python@ -E
      python-script ...
    \item Everything that depends on PATH must be hard coded
    \item basename $\Rightarrow$ /bin/basename
    \item Unique install for each operating system.
    \item Programs move around: basename
  \end{itemize}
\end{frame}

\begin{frame}{Using XALT Data}
  \begin{itemize}
    \item Targetted Outreach: Who will be affected
    \item Largemem Queue Overuse
    \item XALT and TACC-Stats
  \end{itemize}
\end{frame}

\begin{frame}{Publishing XALT Data}
  \begin{itemize}
    \item Student Sandra Sweat
    \item Sanitized Data
    \item Community Codes Reported: Vasp*, WRF*, OpenFOAM*,
    \item users names : U012354, Charge Accounts: A12345
    \item Unique mapping, Added Field of Science
  \end{itemize}
\end{frame}

\begin{frame}{Tracking Non-mpi jobs (I)}
  \begin{itemize}
    \item Originally we tracked only MPI Jobs
    \item By hijacking mpirun etc.
    \item Now we can use ELF binary format to track jobs
  \end{itemize}
\end{frame}

\begin{frame}[fragile]
    \frametitle{ELF Binary Format Trick}
 {\small
    \begin{alltt}
void myinit(int argc, char **argv)
\{
  /* ... */
\}
void myfini()
\{
  /* ... */
\}
  __attribute__((section(".init_array")))
       typeof(myinit) *__init = myinit;
  __attribute__((section(".fini_array")))
       typeof(myfini) *__fini = myfini;
    \end{alltt}
}
\end{frame}

\begin{frame}{Using the ELF Binary Format Trick}
  \begin{itemize}
    \item This C code is compiled and linked in through the hijacked linker
    \item It can also be used with \texttt{LD\_PRELOAD}
    \item We are using both...
  \end{itemize}
\end{frame}

\begin{frame}{Downsides}
  \begin{itemize}
    \item Currently, we only track task 0 jobs.
    \item MPMD programs will only record the Task 0 job.
    \item We also lose the ability to capture return exit status
  \end{itemize}
\end{frame}

\begin{frame}{Upsides (I)}
  \begin{itemize}
    \item Can now track all executables period.
    \item Can now track ``launcher'' jobs.
  \end{itemize}
\end{frame}

\begin{frame}{Upsides (II)}
  \begin{itemize}
    \item Do not need to write/maintain a parser for ibrun, mpirun ...
    \item Do not need to correctly jump over certain executables:
    \begin{itemize}
      \item OK: \texttt{ibrun tacc\_affinity} \emph{user\_program}
      \item Not O.K: \texttt{ibrun env -u foo} \emph{user\_program}
    \end{itemize}
  \end{itemize}
\end{frame}

\begin{frame}{Challenges (I)}
  \begin{itemize}
    \item With both \texttt{LD\_PRELOAD} and init.o linked
      in. $\Rightarrow$ double records
    \item Do not want to track mv, cp, etc
    \item Only want to track some executables on compute nodes
    \item Do not want to get overwhelmed by the data. 
  \end{itemize}
\end{frame}

\begin{frame}{Why do both?}
  \begin{itemize}
    \item We want both linking in and \texttt{LD\_PRELOAD}, Why?
      \begin{itemize}
        \item Data on programs built before XALT
        \item Data on GUI debugger, ...
        \item User sets \texttt{LD\_PRELOAD}
      \end{itemize}
  \end{itemize}
\end{frame}

\begin{frame}{Avoid Double counting}
  \begin{itemize}
    \item \texttt{.init\_array} and \texttt{\.fini\_array} work like
      an onion.
    \item \texttt{.init\_array}: a Stack: LIFO
    \item \texttt{.fini\_array}: a Queue: LILO
    \item Preload, Built-in, program, Built-in, Preload
    \item Use an env. var. to prevent double counting
  \end{itemize}
\end{frame}

\begin{frame}{Other Safety Features}
  \begin{itemize}
    \item XALT Tracking only told to
    \item Compute node only
    \item Filter based on Path
    \item Protection against closing stderr before fini.
  \end{itemize}
\end{frame}

\begin{frame}{Path Filtering}
  \begin{itemize}
    \item Accept test, following an Ignore Test,
    \item Two files containing regex patterns, converted to code.
    \item Accept List Tests: Track /usr/bin/ddt, /bin/tar
    \item Ignore List Tests: /usr/bin, /bin, /sbin, ...
  \end{itemize}
\end{frame}

\begin{frame}{A LD\_PRELOAD debug version}
  \begin{itemize}
    \item Normal Version is fast with minimal tests.
    \item A debug version is provide to help with testing:
    \item LD\_PRELOAD=\$XALT\_DEBUG\_INIT ./a.out
  \end{itemize}
\end{frame}

\begin{frame}{XALT Demo}
  \begin{itemize}
    \item Show modules hierarchy
    \item ml --raw show xalt
    \item Show debugging output
    \item type -a {ld,mpirun}
    \item Build programs
    \item Run tests
    \item Run utf8 program
    \item Show database results
  \end{itemize}
\end{frame}


\section{Conclusion}
\begin{frame}{Conclusion}
  \begin{itemize}
    \item Lmod:
      \begin{itemize}
        \item Source: github.com/TACC/lmod.git, lmod.sf.net
        \item Documentation: lmod.readthedocs.org
      \end{itemize}
    \item XALT:
      \begin{itemize}
        \item Source: github.com/Fahey-McLay/xalt.git, xalt.sf.net
        \item Documentation: doc/*.pdf
      \end{itemize}
  \end{itemize}
\end{frame}

\end{document}
