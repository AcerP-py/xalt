% Due AoE Feb 15, 2015


\documentclass[12pt]{article}
%\usepackage{diss_inc}
%\usepackage{psfig}
\usepackage{underscore}
\setlength{\textwidth}{6.5in}
\setlength{\textheight}{9.0in}
\setlength{\topmargin}{0.0in}
\setlength{\headheight}{0.0in}
\setlength{\headsep}{0.0in}
\setlength{\oddsidemargin}{0in}
\setlength{\evensidemargin}{0in}
\begin{document}

%\noindent
%{\bf{}Title:} Job-Level Tracking with XALT: A Tutorial for System Administrators and Data Analysts
%
%~~
%
%
%\noindent
%{\bf{}Abstract:}
%
%
%% 200 words
%
%Let’s talk real, no-kiddin’ supercomputer analytics, aimed at moving
%beyond monitoring the machine as a whole or even its individual
%hardware components. We’re interested in drilling down to the level of
%individual batch submissions, users, and binaries. We’re after ready
%answers to the "what, where, how, when and why" that stakeholders are
%clamoring for – everything from which libraries (or individual
%functions!) are in demand to preventing the problems that get in the
%way of successful science. This tutorial will show how to install and
%set up the XALT tool that can provide this type of job-level insight.
% 
%The XALT tool can provide a wide range of metrics and measures of
%job-level activity. There are benefits to users and stakeholders:
%sponsoring institutions interested in strategic priorities;
%organizations concerned about meeting users’ needs; and those seeking
%to study user activity to improve value and effectiveness.
% 
%We will show how to install and configure XALT. We will also show how
%this tool provides high value to centers and their users as it can
%provide documentation on how an application was built to provide
%reproducibility. 
%
%~~
%
%\noindent
%{\bf{}Topic Area:}
%
%Monitoring \& administration tools
%
%~~
%
%\noindent
%{\bf{}Keywords:} 
%
%Reproducibility,
%User environments,
%Job Level Monitoring,
%Library usage
%~~

\noindent
{\bf{}Title:} 


Drilling Down: Understanding User-Level Activity on Today's Supercomputers With XALT

~~

\noindent
{\bf{}Topic:}
This BoF will bring together those with experience and interest in
present and future system tools and technologies that can provide
library and application usage and job--level monitoring type of
job--level insight, and will be the kickoff meeting for a new Special
Interest Group for those who want to explore this topic more deeply.

~~

\noindent
{\bf{}Estimated Attendance:} 
30-40  (Previous BoF at	ISC-14 had 45, a similar BoF at	SC had 48 and
XSEDE-14 had 30.)	


~~

\noindent
{\bf{}Short CV:} 

~~

\noindent
{\bf{}Dr. Mark Fahey}

~~

\noindent
Dr. Mark R. Fahey is the Director of Operations for Argonne National
Labs and holds a joint faculty position at the University of
Chicago. Formally, he was a Deputy Director at NICS, and helped run one of
the National Science Foundation’s supercomputing centers.  Dr. Fahey
was also the lead for the Extended Support for Research Teams component
of the XSEDE project.  He received his B.A. From St. Norbert
College in 1992 and his Ph.D. from the University of Kentucky in 1999.  

Dr. Fahey has interests in the areas of numerical algorithms, code
optimization techniques, parallel performance and scalability, and in
software management techniques and tools.  In particular, Dr. Fahey
has led the effort of the development of an Automatic Library Tracking
Database tool that is in production at various centers around the
globe and has recently won an NSF grant to do further research and
development in this area. 

~~

\noindent
{\bf{}Dr. Robert McLay}

~~

\noindent
Dr. Robert McLay received bachelors and masters degree from the Massachusetts 
Institute of Technology and his Ph.D in Engineering Mechanics from The University 
of Texas at Austin.  His research include C++ software development, regression 
testing, and software tools, all related to large parallel numerical simulation codes.  
In particular, he has done work in parallel finite-element programs solving 
incompressible fluid flow and heat transfer.

His interest in software tools and support of HPC programming environments has 
lead to his development of Lmod, a modern replacement for Environment Modules 
system.  Lmod's major advantages are protecting all users from loading incompatible 
software without hindering experts.  This work has lead to an interest in tracking the 
software usage through the module system.

\end{document}
