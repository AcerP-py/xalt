% Due AoE Feb 15, 2015


\documentclass[12pt]{article}
%\usepackage{diss_inc}
%\usepackage{psfig}
\usepackage{underscore}
\setlength{\textwidth}{6.5in}
\setlength{\textheight}{9.0in}
\setlength{\topmargin}{0.0in}
\setlength{\headheight}{0.0in}
\setlength{\headsep}{0.0in}
\setlength{\oddsidemargin}{0in}
\setlength{\evensidemargin}{0in}
\begin{document}

%\noindent
%{\bf{}Title:} Job-Level Tracking with XALT: A Tutorial for System Administrators and Data Analysts
%
%~~
%
%
%\noindent
%{\bf{}Abstract:}
%
%
%% 200 words
%
%Let’s talk real, no-kiddin’ supercomputer analytics, aimed at moving
%beyond monitoring the machine as a whole or even its individual
%hardware components. We’re interested in drilling down to the level of
%individual batch submissions, users, and binaries. We’re after ready
%answers to the "what, where, how, when and why" that stakeholders are
%clamoring for – everything from which libraries (or individual
%functions!) are in demand to preventing the problems that get in the
%way of successful science. This tutorial will show how to install and
%set up the XALT tool that can provide this type of job-level insight.
% 
%The XALT tool can provide a wide range of metrics and measures of
%job-level activity. There are benefits to users and stakeholders:
%sponsoring institutions interested in strategic priorities;
%organizations concerned about meeting users’ needs; and those seeking
%to study user activity to improve value and effectiveness.
% 
%We will show how to install and configure XALT. We will also show how
%this tool provides high value to centers and their users as it can
%provide documentation on how an application was built to provide
%reproducibility. 
%
%~~
%
%\noindent
%{\bf{}Topic Area:}
%
%Monitoring \& administration tools
%
%~~
%
%\noindent
%{\bf{}Keywords:} 
%
%Reproducibility,
%User environments,
%Job Level Monitoring,
%Library usage
%~~

\noindent
{\bf{}Detailed Description:} 

% Overview and goals of the tutorial (takeaways for the audience)
% Detailed outline of the tutorial
% URLs to sample slides and other material

This tutorial will show how to install and configure XALT – a tool for
drilling down to the level of individual batch submissions, users, and
binaries. We’re after ready answers to the "what, where, how, when and
why" that stakeholders are clamoring for – everything from which
libraries (or individual functions!) are in demand to preventing the
problems that get in the way of successful science. We will show how
the XALT tool can be used to provide attention across a wide range of
metrics and measures of job-level activity. We see potential benefits
to stakeholders beyond just end users: sponsoring institutions
interested in strategic priorities and scientific impact; support
organizations and development teams concerned about meeting users’
needs and expectations; and those seeking to study user activity to
improve value and effectiveness. We will also show how XALT provides
high value to users as it can provide complete documentation on how an
application was built and when it was run; whenever you want to figure
it out again. This provides a key piece of the reproducibility
needs. We will also discuss how this can be a valuable tool for
security reasons again due to the tracking of shared (and static)
libraries in an application over time. 

The proposers bring to the table passion and experience improving the
end-user experience. Both Fahey and McLay are actively engaged in work
focused on job-level activity. Dr. Fahey is the author of ALTD, a tool
that reports software and library usage at the individual job
level. Dr. McLay is the author of Lmod, an innovative environmental
module system with numerous features that facilitate job-level
analysis and protect the user from common configuration problems. Both
tools are currently deployed at numerous major centers across the
United States and Europe. 

The expected outline with time estimations are as follows:
Introduction and Motivation, 20 min Prepare to install, 20 min -
Things to consider Installation, 50 min - Configure, make, install -
Post install changes Break, 15 min Testing, 30 min - Debugging
Production, 30 min - Wrappers - Data transmission Data mining, 40 min 

\noindent
Sample slides:

https://sourceforge.net/projects/xalt/files/Presentations/xalt-tutorial-slides.pdf 

~~

\noindent
{\bf{}Logistics:}

This is a half-day tutorial.  The material is broken down into 20\%
Beginner, 50\% Intermediate and 30\% Advanced.   Attendees should
bring a laptop to get the full benefit of the tutorial.

~~

\noindent
{\bf{}Short CV:} 

\noindent
{\bf{}Dr. Mark Fahey}

~~

\noindent
Dr. Mark R. Fahey is the Director of Operations for Argonne National
Labs and Joint Faculty at the University of Chicago; and Dr. Fahey is
part of the Scientific Computing Group. Formally was Deputy directory 
at the National Center for Computational Sciences at Oak Ridge
National Laboratory.  As Deputy Director at NICS, Dr. Fahey helped run one of
the National Science Foundation’s supercomputing centers.  Dr. Fahey
is also the lead for the Extended Support for Research Teams component
of the XSEDE project.  Dr. Fahey received his B.A. From St. Norbert
College in 1992 and his Ph.D. from the University of Kentucky in 1999.  

Dr. Fahey has interests in the areas of numerical algorithms, code
optimization techniques, parallel performance and scalability, and in
software management techniques and tools.  In particular, Dr. Fahey
has led the effort of the development of an Automatic Library Tracking
Database tool that is in production at various centers around the
globe and has recently won an NSF grant to do further research and
development in this area. 

~~

\noindent
{\bf{}Dr. Robert McLay}

~~

\noindent
Dr. Robert McLay received bachelors and masters degree from the Massachusetts 
Institute of Technology and his Ph.D in Engineering Mechanics from The University 
of Texas at Austin.  His research include C++ software development, regression 
testing, and software tools, all related to large parallel numerical simulation codes.  
In particular, he has done work in parallel finite-element programs solving 
incompressible fluid flow and heat transfer.

His interest in software tools and support of HPC programming environments has 
lead to his development of Lmod, a modern replacement for Environment Modules 
system.  Lmod's major advantages are protecting all users from loading incompatible 
software without hindering experts.  This work has lead to an interest in tracking the 
software usage through the module system.

\noindent
%{\bf{}Travel funding request:}

%Yes for both speakers.

\end{document}
